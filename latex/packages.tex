% The input files (tex files) are encoded with the latin-1 encoding 
% (ISO-8859-1 works). Change the latin1-option if you use UTF8 
% (at some point LaTeX did not work with UTF8, but I'm not sure
% what the current situation is) 
\usepackage[latin1]{inputenc}
% OT1 font encoding seems to work better than T1. Check the rendered
% PDF file to see if the fonts are encoded properly as vectors (instead
% of rendered bitmaps). You can do this by zooming very close to any letter 
% - if the letter is shown pixelated, you should change this setting 
% (try commenting out the entire line, for example).  
\usepackage[OT1]{fontenc}
% The babel package provides hyphenating instructions for LaTeX. Give
% the languages you wish to use in your thesis as options to the babel
% package (as shown below). You can remove any language you are not
% going to use.
% Examples of valid language codes: english (or USenglish), british, 
% finnish, swedish; and so on.
\usepackage[finnish,swedish,english]{babel}
\usepackage{csquotes}


% Font selection
% ------------------------------------------------------------------
% The default LaTeX font is a very good font for rendering your 
% thesis. It is a very professional font, which will always be 
% accepted. 
% If you, however, wish to spicen up your thesis, you can try out
% these font variants by uncommenting one of the following lines
% (or by finding another font package). The fonts shown here are 
% all fonts that you could use in your thesis (not too silly). 
% Changing the font causes the layouts to shift a bit; you many
% need to manually adjust some layouts. Check the warning messages
% LaTeX gives you.
% ------------------------------------------------------------------
% To find another font, check out the font catalogue from
% http://www.tug.dk/FontCatalogue/mathfonts.html
% This link points to the list of fonts that support maths, but
% that's a fairly important point for master's theses.
% ------------------------------------------------------------------
% <rant>
% Remember, there is no excuse to use Comic Sans, ever, in any
% situation! (Well, maybe in speech bubbles in comics, but there 
% are better options for those too)
% </rant>

% \usepackage{palatino}
% \usepackage{tgpagella}

% Optional packages
% ------------------------------------------------------------------
% Select those packages that you need for your thesis. You may delete
% or comment the rest.

% Natbib allows you to select the format of the bibliography references.
% The first example uses numbered citations: 
%\usepackage[square,sort&compress,numbers]{natbib}
% The second example uses author-year citations.
% If you use author-year citations, change the bibliography style (below); 
% acm style does not work with author-year citations.
% Also, you should use \citet (cite in text) when you wish to refer
% to the author directly (\citet{blaablaa} said blaa blaa), and 
% \citep when you wish to refer similarly than with numbered citations
% (It has been said that blaa blaa~\citep{blaablaa}).
% \usepackage[square]{natbib}

% I'll use biblatex!
\usepackage[citestyle=ieee,urldate=iso8601]{biblatex}
\DefineBibliographyStrings{english}{urlseen = {accessed on}}

% The alltt package provides an all-teletype environment that acts
% like verbatim but you can use LaTeX commands in it. Uncomment if 
% you want to use this environment. 
% \usepackage{alltt}

% The eurosym package provides a euro symbol. Use with \euro{}
\usepackage{eurosym} 

% Verbatim provides a standard teletype environment that renderes
% the text exactly as written in the tex file. Useful for code
% snippets (although you can also use the listings package to get
% automatic code formatting). 
\usepackage{verbatim}

% The listing package provides automatic code formatting utilities
% so that you can copy-paste code examples and have them rendered
% nicely. See the package documentation for details.
% \usepackage{listings}

% The fancuvrb package provides fancier verbatim environments 
% (you can, for example, put borders around the verbatim text area
% and so on). See package for details.
% \usepackage{fancyvrb}

% Supertabular provides a tabular environment that can span multiple 
% pages. 
%\usepackage{supertabular}
% Longtable provides a tabular environment that can span multiple 
% pages. This is used in the example acronyms file. 
\usepackage{longtable}

% The fancyhdr package allows you to set your the page headers 
% manually, and allows you to add separator lines and so on. 
% Check the package documentation. 
% \usepackage{fancyhdr}

% Subfigure package allows you to use subfigures (i.e. many subfigures
% within one figure environment). These can have different labels and
% they are numbered automatically. Check the package documentation. 
\usepackage{subfigure}

% The titlesec package can be used to alter the look of the titles 
% of sections, chapters, and so on. This example uses the ``medium'' 
% package option which sets the titles to a medium size, making them
% a bit smaller than what is the default. You can fine-tune the 
% title fonts and sizes by using the package options. See the package
% documentation.
\usepackage[medium]{titlesec}

% The TikZ package allows you to create professional technical figures.
% The learning curve is quite steep, but it is definitely worth it if 
% you wish to have really good-looking technical figures. 
\usepackage{tikz}
% You also need to specify which TikZ libraries you use
\usetikzlibrary{positioning}
\usetikzlibrary{calc}
\usetikzlibrary{arrows}
\usetikzlibrary{decorations.pathmorphing,decorations.markings}
\usetikzlibrary{shapes}
\usetikzlibrary{patterns}

% Enumitem for better spacing in lists
\usepackage{enumitem}
\setlist[itemize]{topsep=0.5ex,itemsep=0ex,parsep=0.5ex}
\setlist[enumerate]{topsep=0.5ex,itemsep=0ex,parsep=0.5ex}
\setlist[description]{topsep=1.2ex,itemsep=0.3ex,parsep=0ex,style=multiline,leftmargin=12ex}

% Theorems and definitions
\usepackage{amsthm}
\newtheorem{definition}{Definition}

% Multiple optional argument commands
\usepackage{xargs}

% code listings
\usepackage{listings}
\lstset{
language=C++,
basicstyle=\ttfamily\footnotesize,
keywordstyle=\color{OliveGreen},
commentstyle=\color{gray},
numbers=none,
numberstyle=\tiny,
stepnumber=1,
numbersep=5pt,
backgroundcolor=\color{white},
frame=none,
tabsize=2,
captionpos=b,
breaklines=true,
breakatwhitespace=true,
showstringspaces=false,
showspaces=false,
showtabs=false,
}

% left aligned chapter titles
\usepackage{sectsty}
\chapterfont{\raggedright}
\allsectionsfont{\raggedright}

% Spelling out numbers
\usepackage{refcount}
\usepackage{fmtcount}

% Nicer tables
\usepackage{booktabs}
\usepackage{bigstrut}

% TikZ, and standalone for saving time with TikZ
\usepackage{standalone}
\usepackage{tikz}
\usetikzlibrary{decorations.pathreplacing}
\usetikzlibrary{decorations.markings}
\usetikzlibrary{positioning}
\usetikzlibrary{shapes}
\usetikzlibrary{calc}

% The aalto-thesis package provides typesetting instructions for the
% standard master's thesis parts (abstracts, front page, and so on)
% Load this package second-to-last, just before the hyperref package.
% Options that you can use: 
%   mydraft - renders the thesis in draft mode. 
%             Do not use for the final version. 
%   doublenumbering - [optional] number the first pages of the thesis
%                     with roman numerals (i, ii, iii, ...); and start
%                     arabic numbering (1, 2, 3, ...) only on the 
%                     first page of the first chapter
%   twoinstructors  - changes the title of instructors to plural form
%   twosupervisors  - changes the title of supervisors to plural form
\usepackage[doublenumbering]{aalto-thesis}

% Hyperref
% ------------------------------------------------------------------
% Hyperref creates links from URLs, for references, and creates a
% TOC in the PDF file.
% This package must be the last one you include, because it has
% compatibility issues with many other packages and it fixes
% those issues when it is loaded.   
\RequirePackage[pdftex]{hyperref}
% Setup hyperref so that links are clickable but do not look 
% different
\hypersetup{colorlinks=false,raiselinks=false,breaklinks=true}
\hypersetup{pdfborder={0 0 0}}
\hypersetup{bookmarksnumbered=true}
% The following line suggests the PDF reader that it should show the 
% first level of bookmarks opened in the hierarchical bookmark view. 
\hypersetup{bookmarksopen=true,bookmarksopenlevel=1}
% Hyperref can also set up the PDF metadata fields. These are
% set a bit later on, after the thesis setup.   
