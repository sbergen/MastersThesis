\chapter{Source Code Availability and Compatibility}

At the time of publishing,
the source code for the implementation is freely
available on \textit{GitHub},
at \url{https://github.com/sbergen/ConductorFollower}.
The project was developed using
\textit{Microsoft Visual Studio 2010} (VS2010)
on \textit{Windows 7}.
The implementation uses some of the C++11 features
available in VS2010, but does not by default use any
Windows specific features.
Thus it should be rather trivial to port
the code to any other platform with:
\begin{itemize}
\item A compiler that supports the used C++11 features,
\item support for OpenNI,
\item support for Juce, and
\item support for Boost.
\end{itemize}
Porting will, however,
require creating a build script for the given environment,
as only VS2010 project files are included.
Feasible platforms include at least
\textit{OS X} and \textit{Linux}.
Pull requests for compatibility fixes for different platforms
are highly welcome.

Sample instrument patches for VSL and their
respective instrument definition files,
along with a beat pattern definition file
can be found from the \texttt{data\_content}
directory in the repository.
The instrument patches require
the \textit{Vienna Special Edition}
volumes 1, 1+, 2, and 2+.

\chapter{DSL Examples}
\label{appendix:dsl_samples}

\section{Score Description Format}

\lstinputlisting{listings/score.def}

\section{Instrument Definition Format}

\lstinputlisting{listings/instruments.def}

\section{Beat Pattern Definition Format}

\lstinputlisting{listings/beat_patterns.def}
