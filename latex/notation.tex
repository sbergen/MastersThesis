
\begin{longtable}{@{}p{0.25\textwidth}p{0.75\textwidth}@{}}

\tuple{\alpha, \beta, \gamma} & Denotes a tuple containing the values $\alpha$, $\beta$, and $\gamma$.
\\

\tupledef{T}{a, b, c} &
Defines the tuple type $T$, which consists of the variables $a$, $b$, and $c$.
\\

\set{\alpha, \beta, \gamma} &
Denotes the set containing the values $\alpha$, $\beta$, and $\gamma$.
\\

\setdef{T}{\alpha, \beta, \gamma} &
Defines the type T, which may only contain values from the set \set{\alpha, \beta, \gamma}.
\\

$\mathbf{a}_i$ &
Denotes the \nth{$i$} element in the array $\mathbf{a}$.
\\

\size{\mathbf{a}} &
Denotes the size of array $\mathbf{a}$.
\\

$\mathcal{F} : D_1 \rightarrow D_2$ &
Denotes that the function $\mathcal{F}$ maps values from domain $D_1$ to domain $D_2$.
\\

\modulo{\alpha}{\beta} &
Denotes the floating point modulus operation. That is, the value $\alpha - N \beta$ for an integer $N$ such that the magnitude of the result is less than $\beta$.
\\

$\lfloor \alpha \rfloor$ &
Denotes the floor function of $\alpha$. That is,
the largest integer smaller than or equal to $\alpha$.
\\

\sgn{\alpha} &
Denotes the signum function of $\alpha$.
\\

\oointerval{\alpha, \beta} &
Denotes an open interval from $\alpha$ to $\beta$.
\\

\ccinterval{\alpha, \beta} &
Denotes a closed interval from $\alpha$ to $\beta$.
\\

\ocinterval{\alpha, \beta} and \cointerval{\alpha, \beta} &
Denote half-open intervals from $\alpha$ to $\beta$.

\end{longtable}